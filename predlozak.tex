\documentclass[conference]{IEEEtran}
\IEEEoverridecommandlockouts
% The preceding line is only needed to identify funding in the first footnote. If that is unneeded, please comment it out.
\usepackage{cite}
\usepackage{amsmath,amssymb,amsfonts}
\usepackage{algorithmic}
\usepackage{graphicx}
\usepackage{textcomp}
\usepackage{xcolor}
\usepackage[T1]{fontenc}
\usepackage[utf8]{inputenc}
\usepackage[croatian]{babel}
\def\BibTeX{{\rm B\kern-.05em{\sc i\kern-.025em b}\kern-.08em
    T\kern-.1667em\lower.7ex\hbox{E}\kern-.125emX}}
\usepackage{hyperref}  % Stavljeno ispod \documentclass
\begin{document}

\title{Paralelno aproksimativno pretraživanje uzoraka u tekstu\\
{\footnotesize \textsuperscript{}Paralelizam i konkurentnost - grupa 15}
\thanks{Fakultet elektrotehnike i računarstva}
}

\author{\IEEEauthorblockN{\textsuperscript{} Eno Peršić}
\IEEEauthorblockA{\textit{Fakultet elektrotehnike i računarstva} \\
\textit{Sveučilište u Zagrebu}\\
Zagreb, Hrvatska\\
eno.persic@fer.hr}
\and
\IEEEauthorblockN{\textsuperscript{} Karla Pišonić}
\IEEEauthorblockA{\textit{Fakultet elektrotehnike i računarstva} \\
\textit{Sveučilište u Zagrebu}\\
Zagreb, Hrvatska \\
karla.pisonic@fer.hr}
\and
\IEEEauthorblockN{\textsuperscript{} Antonio Sabljić}
\IEEEauthorblockA{\textit{Fakultet elektrotehnike i računarstva} \\
\textit{Sveučilište u Zagrebu}\\
Zagreb, Hrvatska \\
antonio.sabljic@fer.hr}
\and
\IEEEauthorblockN{\textsuperscript{} Domagoj Sviličić}
\IEEEauthorblockA{\textit{Fakultet elektrotehnike i računarstva} \\
\textit{Sveučilište u Zagrebu}\\
Zagreb, Hrvatska \\
domagoj.svilicic@fer.hr}
\and
\IEEEauthorblockN{\textsuperscript{} Hrvoje Tkalčević}
\IEEEauthorblockA{\textit{Fakultet elektrotehnike i računarstva} \\
\textit{Sveučilište u Zagrebu}\\
Zagreb, Hrvatska \\
hrvoje.tkalcevic@fer.hr}
}

\maketitle

\begin{abstract}
Ovaj dokument predstavlja dio projektne dokumentacije u sklopu projektnog zadatka u okviru kolegija Paralelizam i konkurentnost na Fakultetu elektrotehnike i računarstva u Zagrebu.

IME PROJEKTA: Usporedba vremena izvođenja algoritama za aproksimativno pretraživanje uzoraka u tekstu s obzirom parametar tolerancije 
\end{abstract}

\begin{IEEEkeywords}
paralelizam, bioinformatika, sekvenciranje genoma
\end{IEEEkeywords}

\section{Uvod}
Aproksimativno pretraživanje uzoraka u tekstu odnosi se na proces pronalaženja uzorka ili podniza unutar teksta uz određenu dopuštenu razinu pogrešaka ili razlika. Taj postupak  koristi se kada je potrebno pronaći uzorak ili slične uzorke unutar teksta, čak i kada nema potpune podudarnosti s uzorkom. 
Primjena aproksimativnog pretraživanja uzoraka je ključna u mnogim područjima, uključujući bioinformatiku (sekvenciranje genoma, pretraživanje sličnih sekvenci), pretraživanje teksta (u pretraživačima, obradi prirodnog jezika, analizi podataka), kompresiji podataka (prepoznavanje i eliminacija ponavljajućih uzoraka), obradi slika i mnogim drugim područjima.  
Ovaj projekt obrađuje problematiku traženja genske sekvence unutar većih referentih nizova koji predstavljaju očitanja uređaja za sekvenciranje. Referentni niz i tražena sekvenca koju je potrebno pronaći kao podniz unutar spomenutog referentnog niza predstavljeni su kao niz nukleotidnih baza A, T, C i G. Nekoliko je razloga zbog čega je primjerenije koristiti aproksimativno prepoznavanje uzoraka u nizovima nukleotida umjesto određivanja potpuno točnih sekvenci:  
\begin{itemize}
\item \textbf{Pogreške u sekvenciranju}: procesi sekvenciranja genoma nisu savršeni i mogu dovesti do pogrešaka. To može uključivati zamjene, umetanja ili brisanja nukleotida. Aproksimativno prepoznavanje uzoraka može pomoći u identifikaciji sekvenci koje su blizu traženog uzorka, unatoč tim pogreškama. 
\item \textbf{Genetske varijacije}: genetski materijal može varirati između različitih jedinki iste vrste. Na primjer, ljudski genomi se razlikuju jedan od drugog za oko 0,1\%. Aproksimativno prepoznavanje uzoraka može pomoći u identifikaciji sekvenci koje su slične traženom uzorku, ali imaju male varijacije. 
\item \textbf{Evolucija i mutacije}: tijekom vremena, genetski materijal može evoluirati i mutirati što može dovesti do promjena u sekvencama nukleotida. To je osobito izraženo kod genoma virusa i bakterija koji se uostalom i najčešće proučavaju unutar područja bioinformatike. Aproksimativno prepoznavanje uzoraka može pomoći u identifikaciji sekvenci koje su evoluirale ili mutirale od traženog uzorka. 
\item \textbf{Funkcionalne varijacije}: u nekim slučajevima, različite sekvence nukleotida mogu imati slične funkcije. Na primjer, različite sekvence mogu kodirati isti protein. Aproksimativno prepoznavanje uzoraka može pomoći u identifikaciji tih funkcionalno sličnih sekvenci. 
\end{itemize}
Jedan od središnjih pojmova u kontekstu aproksimativnog pretraživanja uzorka u tekstu predstavlja  parametar tolerancije koji se definira kao dopušteni broj grešaka ili razlika koji se uzima u obzir prilikom usporedbe traženog uzorka s podacima u tekstu. Ova tolerancija omogućuje algoritmima koji rade s podacima sklonim pogreškama da pronađu uzorak ili podniz unutar teksta, čak i ako nije potpuno identičan traženom uzorku. Ovaj projekt bazira se na proučavanju ovisnosti vremena izvođenja različitih algoritama za aproksimativno pretraživanje uzoraka u tekstu o iznosu parametra tolerancije. U osnovi, veći broj dopuštenih pogrešaka očito će povećati vrijeme potrebno za pretragu jer algoritam mora provjeriti više mogućih podudaranja. Međutim, točan utjecaj može se razlikovati ovisno o specifičnom algoritmu i načinu na koji se pogreške obrađuju. Korišteni algoritmi su Levenshteinov algoritam, Rabin-Karp, naivni algoritam i Bitap algoritam. Međusobno se razlikuju po pomoćnim strukturama koje koriste te načinima kojima sužavaju prostor rješenja. Svi ovi algoritmi s većim ili manjim modifikacijama nalaze  primjenu kako unutar područja bioinformatike tako i izvan njega. U ovome projektu osnovna im je namjena odabir podnizova iz referentnog niza koji se od traženog uzorka razlikuju za iznos definiran parametrom tolerancije. Takvi kandidati traženog uzorka potom se najčešće koriste kao grupa nad kojom se mogu učinkovitije vršiti bilo kakve daljnje analize. Različiti algoritmi ne moraju nužno prepoznati isti skup podnizova kao kandidate jer svaki koristi neku metodu kodiranja ili sažimanja kojom u svrhu ostvarivanja boljih  performansi gubi dio potencijalnih rješenja. Važno je da svaki stvori skup kandidata koji će biti operabilan za daljnje analize preciznijim algoritmima. Cilj projekta je da prosljeđivanjem  konstatnog traženog uzorka nad konstantnim skupom referentnih nizova i variranjem parametra tolerancije otkrijemo koji od korištenih algoritama nudi najveću robusnost, tj. najkonstantnije vrijeme izvođenja. Obzirom da parametar tolerancije nije broj koji je moguće jednostavno izračunati, nego se najčešće određuje empirijski  ovisno o konkretnom problemu, ovaj projekt mogao bi ponuditi odgovor na pitanje koji od algoritama bi predstavljao najbolji izbor za korištenje prilikom postupka traženja tog parametra. 

\subsection{Ulazni podaci}
Polazišna točka projekta je tekstualna datoteka u FASTA formatu koja predstavlja realna očitanja uređaja za sekvencioniranje genoma provedena nad uzorkom bakterije (Escherichia coli K-12 substr. MG1655 genome). U bioinformatici i biokemiji, FASTA format je tekstualni format za prikazivanje nukleotidnih sekvenci ili sekvenci aminokiselina, u kojima se nukleotidi ili aminokiseline prikazuju pomoću jednoslovnih kodova. Format omogućuje da imena sekvenci i komentari prethode sekvencama. Potekao je iz softverskog paketa FASTA i od tada je postao gotovo univerzalni standard u bioinformatici. 

Jednostavnost FASTA formata olakšava manipulaciju i parsiranje sekvenci pomoću alata za obradu teksta i skriptnih jezika. Sekvencija započinje znakom veće od (“>”) praćenim opisom sekvence (sve u jednom retku). Linije koje odmah slijede liniju opisa su prikaz sekvence, s jednim slovom po aminokiselini ili nukleinskoj kiselini. Dotična datoteka dostupna je na sljedećem linku \url{https://nanopore.s3.climb.ac.uk/MAP006-1_2D_pass.fasta} i preuzeta je od istraživačke skupine Loman Labs sa Sveučilišta u Birminghamu (detaljnije na \url{http://lab.loman.net/2015/09/24/first-sqk-map-006-experiment/}). Svaka sekvenca očitanja potpuno je nezavisna od drugima u smislu da se ne mora nadovezivati na prethodnu i sljedeću unutar datoteke te da njen položaj u datoteci ne preslikava njen položaj u stvarnom genomu. To je posljedica činjenice da uređaji za sekvencioniranje nisu sposobni očitati cijeli genom nekog uzorka već samo sekvence ili dijelove tog genoma. Na temelju tih očitanja  i njihovih međusobnih preklapanja (ako ih je provedno dovoljno veliki broj) moguće je određenim tehnikama rekonstruirati cijeli genom mikroorganizma. Cjelokupni genom bakterije Echerichije coli dostupan je na stranicama Oxford Nanopor Technologies (sa linka: \url{https://ftp.ncbi.nlm.nih.gov/genomes/all/GCF/000/005/845/GCF_000005845.2_ASM584v2/GCF_000005845.2_ASM584v2_genomic.fna.gz}). 
%TODO ovo jos dovrsiti
\subsection{Metodologija}
Tu napišite proces kreiranja datoteke, load balancinga i algoritme koje smo koristili.
Koristili smo paralelizaciju s 8 dretvi jer smo se vodili činjenicom da testno računalo ima 8 logičnih jezgri pa se na svakoj jezgri može nesmetano izvoditi jedna dretva i to bi trebalo pružiti najbolje performanse izvođenja.\\
Prije provođenja samog testiranja ponašanja algoritama te konačnog zaključka, važno je razumijeti kako koji algoritam radi kako bi se moglo pokušati predvidjeti djelovanje svakog od njih. Prvi je algoritam Levenshteinov algoritam koji svoju funkcionalnost bazira na Levensteinovoj udaljenosti između dva znakovna niza. Ta se udaljenost računa kao broj znakova koji se mora promijeniti između dva stringa da bi ona postala identična. Na primjer, za nizove ATCG i ACTG Levensteinova je udaljenost 2, dok je za nizove ATCG i TCTG 1, jer je to broj znakova koje treba promijeniti kako bi nizovi postali jednaki. Sljedeći je testirani algoritam vjerojatno najpoznatiji algoritam za traženje podudaranja među znakovnim nizovima nazvan Rabin-Karp po njegovim autorima. Glavna funkcija usporedbe podudaranja dva znakovna niza sada je \textit{rolling hash} funkcija koja svakom podnizu originalnog stringa pridjeljuje određenu brojčanu vrijednost te se na temelju toga broja uspoređuje. Funkcija je "kontrljajuća" jer se temelji na prolasku for petlje znak po znak i usporedbi podniza referentnog niza s traženim uzorkom. Vrijednost \textit{hasha} se tako ažurira sa svakim prolaskom kroz petlju, tj. prolaskom po znakovima, pa se može reći da se "kotrlja". Najjednostavniji testirani algoritam nema neko posebno ime pa je samo nazvan naivnim, a temelji se na dvije petlje koje su zadužene za usporedbu referentnog niza te podniza i brojanju nepodudaranja. Ako je taj broj veći od tolerancije, petlja se prekida te se traži idući uzorak.
Posljednji je Bitap algoritam koji se također koristi za učinkovitu pretragu uzorka u tekstu čak i ako postoje male razlike ili pogreške u uzorku. Algoritam funkcionira na način da se  najprije napravi "maska", odnosno dodjeljuje se jedinstveni bit svakom znaku u uzorku, a nakon toga se, na temelju uzorka, stvaraju dvije maske - maska i R što predstavlja obrnutu masku. Izrađuje se i inicijalizira tablica stanja kako bi se pratio proces uparivanja. Algoritam onda obrađuje tekst znak po znak te ažurira stanja na temelju pročitanih znakova. U svakoj se iteraciji provjerava podudaraju li se znakovi na način da se provjerava desni bit u tablici stanja, a ako se podudaraju, to označava uspješnost algoritma.\\
Svi algoritmi na kraju uzimaju odgovarajuće podnizove te ih šalju u kanal pa ih je moguće ispisati. Važno je napomenuti da su algoritmi modificirani kako bi radili s tolerancijom koja se zadaje u pozivu funkcije samog algoritma pa ima smisla ispisivati odgovarajuće pronađene podnizove budući da oni ne moraju biti identični traženoj referenci.

\subsection{Prikaz rezultata i zaključak}
U nastavku su prikazani grafovi ovisnosti vremena izvođenja svakog od korištenih algoritama u odnosu na parametar tolerancije. Obzirom na veliku disipaciju vremena izvođenja između različitih algoritama, grafički prikaz je razdvojen u 3 grafa koji imaju različito limitirane vrijednosti na y-osi kako bi dovoljno dobro mogli pokazati međuodnose između algoritama. Možemo reći kako je GRAF2 uvećana verzija GRAFA1 , a GRAF3 uvećana verzija GRAFA2.
Grafovi su nacrtani pomoću paketa ggplot2 unutar programskog jezika R. Vremena izvođenja za svaki algoritam upisana su .csv datoteku koja je učitana u radnu okolinu i uz pomoć brojnih vizualizacijskih alata koje nam R pruža stvorene su vizualizacije koje bi trebale ponuditi nedvosmislen odgovor na pitanje koji od algoritama bi predstavljao najbolji izbor za korištenje prilikom postupka traženja parametra tolerancije. Najprije je ostvaren točkasti prikaz ovisnosti vremena ovisnosti svakog algoritma o prosljeđenom parametru tolerancije, a potom su kroz te točke provučeni aproksimativni pravci (svaki pravac predstavlja jedan algoritam),a osjenčani dijelovi oko tih pravaca predstavljaju intervale pozdanosti.

\begin{figure}[htbp]
\centerline{\includegraphics[width=0.5\textwidth]{Grafovi_i_csv_datoteke/GRAF1.png}}
\caption{GRAF1 - Ovisnost vremena izvođenja o parametru tolerancije}
\label{fig}
\end{figure}



\begin{figure}[htbp]
\centerline{\includegraphics[width=0.5\textwidth]{Grafovi_i_csv_datoteke/GRAF2.png}}
\caption{GRAF2 - Ovisnost vremena izvođenja o parametru tolerancije}
\label{fig}
\end{figure}

\begin{figure}[htbp]
\centerline{\includegraphics[width=0.5\textwidth]{Grafovi_i_csv_datoteke/GRAF3.png}}
\caption{GRAF3 - Ovisnost vremena izvođenja o parametru tolerancije}
\label{fig}
\end{figure}
\newpage
Odgovor na pitanje: "Koji od korištenih algoritama nudi najveću robusnost, tj. najkonstantnije vrijeme izvođenja prilikom variranja parametra tolerancije" lako ćemo dobiti proučavajući nagibe pravaca. Očito će algoritam sa najpoloženijim pravcem biti odgovor na naše pitanje. 
Na GRAFU1 jasno je vidljivo kako se pravac Levenshteinovog algoritma jako brzo uspinje u odnosu na druge pravce. Metodom eliminacije njega odmah možemo izbaciti iz daljnjeg proučavanja.
To zapravo znači da će brzinja njegova izvođenja najviše ovisiti o toleranciji što i ima smisla ako se proučava implementacija programa. Udaljenost koja se koristi kao uvijet će se računati dok je manja od tolerancije, a budući da je sama funkcija Levensteinove udaljenosti ovdje kompleksnosti O(n*m), gdje su m i n duljine podniza i podniza reference, za veliku će se toleranciju puno puta pozvati ta funkcija. Ostala su još tri algoritma za usporediti.
Kako situacija nad preostalim pravcima nije očito vidljiva promatrajući isključivo GRAF1, stvorili smo grafički prikaz GRAF2 na kojem smo postavili manji limit na y-osi. Sada je opet trivijalno uočiti da graf algoritma Rabin-Karp najbrže raste u odnosu na druge.
Odnos između preostala dva pravca je također već vidljiv, a u GRAFU3 i dodatno je naglašeno kako pravac algoritma Bitap ima najblaži rast te je on pobjednik našeg istraživačkog pitanja. 
Bitap se pokazao najboljim algoritmom zbog korištenja operacija nad bitovima i operacije pomaka što ga čini računalno brzim, a uz to ima i linearnu složenost, pa je njegova vremenska robusnost veća nego kod drugih algoritama.
Testni primjeri napravljeni su na računalu sa modelom procesora: "AMD Ryzen 3 5425U with Radeon Graphics" i operacijskim sustavom Ubuntu 22.04. Eventualne nadogradnje ovog projekta uključivale bi i testiranje na više različitih platformi kako se bolje usporedili dobiveni rezultati.

\begin{thebibliography}{00}
\bibitem{b1} Miller,J.R. et al. (2010) Assembly algorithms for next-generation sequencing data.
Genomics, 95, 315–327.
\bibitem{b2} Ohlebusch,E. \& Gog,S. (2010) Efficient algorithms for the all-pairs suffix-prefix problem and the all-pairs substring-prefix problem. Inf. Process. Lett., 110, 123–128.
\bibitem{b3} Gusfield,D. (1997) Algorithms on strings, trees, and sequences: computer science and
computational biology Cambridge University Press.
\bibitem{b4} Butler,J. et al. (2008) ALLPATHS: de novo assembly of whole-genome shotgun
microreads. Genome Res., 18, 810–20.
\bibitem{b5} Simpson,J.T. et al. (2009) ABySS: a parallel assembler for short read sequence data.
Genome Res., 19, 1117–23.

\end{thebibliography}

\end{document}